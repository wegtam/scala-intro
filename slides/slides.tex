\documentclass[10pt]{beamer}
\usetheme{metropolis}
\usepackage[utf8]{inputenc}
\usepackage{booktabs}
\usepackage{lmodern}
\title{Scala Introduction}
\date{\today}
\author{Jens Grassel}
\institute{Wegtam GmbH}
\begin{document}
  \maketitle

  \section{Environment and Tooling}
  \begin{frame}{Java Virtual Machine}
    Scala runs on the JVM, but not only there!
    \begin{itemize}
      \item Scala.js for Javascript endpoints
      \item Scala Native for LLVM endpoints
    \end{itemize}
  \end{frame}

  \begin{frame}{Build Tools}
    \begin{itemize}
      \item multiple build tools are available
	\begin{itemize}
	  \item SBT
	  \item Mill
	  \item CBT
	  \item Fury
	\end{itemize}
      \item SBT has the biggest ecosystem and is widely used
      \item stick to SBT but maybe checkout Mill ;-)
    \end{itemize}
  \end{frame}

  \begin{frame}{SBT Setup}
    SBT is usually setup on a per project basis via the \texttt{build.sbt} file.
    However here are some nice settings for the global configuration, put it 
    into \texttt{\~{}/.sbt/1.0/global.sbt}\\
    \begin{texttt}
      \\
      // Prevent Strg+C from killing sbt.\\
      cancelable in Global := true\\
      // Coloured console.\\
      initialize \~{}= (\_ $\Rightarrow$\\
      ~~if (ConsoleLogger.formatEnabled)\\
      ~~~~sys.props("scala.color") = "true"\\
      )
    \end{texttt}
  \end{frame}

  \begin{frame}{SBT Plugins}
    Some useful global plugins can be enabled by adding them to the file
    \texttt{\~{}/.sbt/1.0/plugins/plugins.sbt}\\
    \begin{texttt}
      \\
      addSbtPlugin("net.virtual-void" \% "sbt-dependency-graph" \% "0.9.2")\\
      addSbtPlugin("com.timushev.sbt" \% "sbt-updates"          \% "0.4.0")
    \end{texttt}
  \end{frame}

  \begin{frame}{Development Environment}
    There are several options available but beginners should maybe stick
    to a full blown IDE which in the case of Scala means IntelliJ IDEA
    with the Scala plugin.\\
    For the confident:
    \begin{itemize}
      \item VS-Code with scala-metals
      \item Vim or Neovim with vim-scala plugin and optionally scala-metals
      \item Emacs
      \item Atom
      \item possibly others\ldots
    \end{itemize}
  \end{frame}

  \section{Basics}
  \begin{frame}{Values}
    \begin{table}
      \caption{Values types in Scala}
      \begin{tabular}{@{} ll @{}}
	\toprule
	Type & Notes\\
	\midrule
	Immutable & Data structures which cannot be changed.\\
	Mutable & Data structures which can be changed.\\
	Val & A variable that cannot be re-assigned.\\
	Var & A variable that can be re-assigned (changed).\\
	\bottomrule
      \end{tabular}
    \end{table}
  \end{frame}

  \begin{frame}{Values II}
    \begin{table}
      \caption{When to use which type?}
      \begin{tabular}{@{} ll @{}}
	\toprule
	Definition & Notes\\
	\midrule
	Immutable Val & The recommended way to go.\\
	Immutable Var & Okay if used in a local scope.\\
	Mutable Val & Try not to use this but there may be applications.\footnote{However \textbf{if} you do this then \textbf{never} pass the value around!}\\
	Mutable Var & \alert{Never ever do this!}\\
	\bottomrule
      \end{tabular}
    \end{table}
  \end{frame}

  \begin{frame}{Recursion}
    As Scala provides tail recursion you should try to use it if possible.
    \begin{alertblock}{But\ldots}
      Please not simply for the sake of using it!\\
      Evaluate if it is necessary.\footnote{Some algorithm are hard or impossible to do with tail recursion!}
    \end{alertblock}
    \metroset{block=fill}
    \begin{block}{Remember}
      Readability and maintainability trump obscure performance gains every time!
    \end{block}
  \end{frame}

  \begin{frame}{Functions}
    Functions are there to make your life easier!
    \begin{itemize}
      \item Use HOF\footnote{Higher Order Functions}
      \item Use Currying
      \item Use polymorphism
    \end{itemize}
    For additional benefit try to stick with \textbf{pure} functions.
    A pure function:
    \begin{itemize}
      \item is total (an output for every input)
      \item is free of side effects
      \item its output does only depend on its input
    \end{itemize}
  \end{frame}

  \section{Implicits}
  \begin{frame}{What are implicits?}
    An \textbf{implicit} value can be used by the compiler to pass it to any
    function which depends on an implicit parameter of the same type.
  \end{frame}

  \begin{frame}{Use cases}
    \begin{itemize}
      \item reduce boiler plate (implicit parameters)
      \item extend existing types with custom functions (wrapper classes)
      \item convert types implicitly (\alert{Do not do this!})
    \end{itemize}
  \end{frame}

  \begin{frame}{Best practices}
    \begin{itemize}
      \item always specify the type of implicit \texttt{val} or \texttt{def}
      \item do not to use implicits for simple datatypes (primitives)
      \item stick to the naming conventions e.g. \texttt{FooOps} when extending \texttt{Foo}
      \item put extension wrappers into \texttt{syntax} objects
      \item \alert{Do not use implicit conversions!}
    \end{itemize}
  \end{frame}

  \section{Objects, Classes and Traits}
  \begin{frame}{Objects}
    \begin{itemize}
      \item the most simple container format in Scala
      \item basically singletons (in Java world)
      \item no constructor and \textbf{no type parameters}\footnote{This will be become important later.}
      \item can extend one class and one or more traits
      \item start out with objects and \textit{upgrade} later if needed
    \end{itemize}
    \begin{alertblock}{About mixing in traits...}
      Try to avoid mixing in a lot of traits into your objects. See the infamous \textit{Cake Pattern} which will lead to tight coupling and other issues.
    \end{alertblock}
  \end{frame}

  \begin{frame}{Classes}
  \end{frame}

  \begin{frame}{Traits}
  \end{frame}

  \section{Type Classes}
  \begin{frame}{Use cases}
  \end{frame}

  \begin{frame}{Encodings}
  \end{frame}

  \begin{frame}{Laws}
  \end{frame}

  \section{Useful libraries for functional programming}
  \begin{frame}{Cats}
  \end{frame}

  \begin{frame}{Refined}
  \end{frame}

  \begin{frame}{ScalaCheck}
  \end{frame}
\end{document}
